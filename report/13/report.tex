本課題では自動音声認識でテキスト化された発話データから
英語学習者の調音ミスを検出するシステムを 提案する.

これは論文です.

\section{はじめに}

この文書は,ごく基本的なレポートや論文の例を示すものです.
実際にこのソースを入力してタイプセット(コンパイル)し,
タイトル,著者名,本文,見出し,箇条書きがどのように表示されるかを
確認してみましょう.

\section{見出し}

この文書の先頭にはタイトル,著者名,日付が出力されています.
特定の日付を指定することもできます.

そして,セクションの見出しが出力されています.
セクションの番号は自動的に付きます.

\section{箇条書き}

\textipa{[""Ekspl@"neIS@n]}

\cite{kato1985,yoshida2008,nakatani2009}.

わからん

\newpage
